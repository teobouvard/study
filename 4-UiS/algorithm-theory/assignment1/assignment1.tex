\documentclass[a4paper, 10pt, twoside]{article}
\usepackage[left=2cm, right=2cm, top=2cm, bottom=3cm]{geometry}
\usepackage{amsmath}
\usepackage[shortlabels]{enumitem}
\usepackage{bbold}
\usepackage{cases}
\usepackage{systeme}
\usepackage{graphicx}

\begin{document}



\title{Algorithm Theory - Assignment 1}
\author{T\'eo Bouvard}
\maketitle

\section*{Problem 1}
In this problem, the objective is to maximize the profit of selling products X and Y, while satisfying production constraints on these products. The profit can be computed as

\begin{align*}
    profit & = revenue - cost                                                                              \\
           & = revenue - (time_{machine} \times cost_{machine} + time_{craftsman} \times cost_{craftsman})
\end{align*}

We can compute profits for each of the products

\begin{align*}
    profit(X) = 200 - (\frac{15}{60} \times 100 + \frac{20}{60} \times 20) = \frac{505}{3} \\
    profit(Y) = 300 - (\frac{20}{60} \times 100 + \frac{30}{60} \times 20) = \frac{770}{3}
\end{align*}

And formulate the problem as a Linear Programming problem. Let $n_X$ and $n_Y$ be the number of products X and Y produced.

\begin{align*}
     & \text{maximize } n_X \times profit(X) + n_Y \times profit(Y) \\
     & \text{subject to }
    \begin{cases}
        n_X \times time_{machine}(X) + n_Y \times time_{machine}(Y) \le 40 \times 60     \\
        n_X \times time_{craftsman}(X) + n_Y \times time_{craftsman}(Y) \le 35 \times 60 \\
        n_X \ge 10                                                                       \\
        n_X, n_Y \ge 0                                                                   \\
    \end{cases}
\end{align*}

Which can be simplified as

\begin{align*}
     & \text{maximize } n_X \times \frac{505}{3} + n_Y \times \frac{770}{3} \\
     & \text{subject to }
    \begin{cases}
        15 \times n_X + 20 \times n_Y \le 2400 \\
        20 \times n_X + 30 \times n_Y \le 2100 \\
        n_X \ge 10                             \\
        n_Y \ge 0                              \\
    \end{cases}
\end{align*}

\section*{Problem 2}
\begin{enumerate}[a)]
    \item

          Let $n_{A}$ and $n_{B}$ be the number of products A and B produced.

          \begin{align*}
               & \text{maximize } n_{A} \times 3 + n_{B} \times 5 \\
               & \text{subject to }
              \begin{cases}
                  12 \times n_{A} + 25 \times n_{B} \le 30 \times 60 \\
                  2 \times n_{B} - 5 \times n_{A} \ge 0              \\
                  n_{A}, n_{B} \ge 0                                 \\
              \end{cases}
          \end{align*}

          \begin{center}
              \includegraphics[width = .5 \textwidth]{graph.png}
          \end{center}

          By graphing the feasible region, we see three candidate points. The first one $I_0 = (0, 0)$ can be trivially discarded as it leads to a profit of 0\$. Let's evaluate the objective function at the two other points. At $I_1 = (72, 0)$ the profit is 216\$. To compute the coordinates of the last point, we solve the following system.

          \begin{align*}
              \systeme{
                  12 \times n_{A} + 25 \times n_{B} = 1800,
                  2 \times n_{B} - 5 \times n_{A} = 0
              }
              \implies
              \systeme{
                  22 \times n_{A} = 1800,
                  n_{B} = \frac{2}{5} \times n_{A}
              }
              \implies
              \systeme{
                  n_{A} = 81.8,
                  n_{B} = 32.7
              }
          \end{align*}

          and at $I_2 = (81.8, 32.7)$ the profit is 408.9\$ which is the highest profit in this case. However, $n_A$ and $n_B$ are not integers. It is not specified in the exercise whether they should be or not, but as the problem is about a weekly production, we can assume that they do not need to be integer quantities as the weeks are continuous. If we wanted to optimal integer solution, we should pick $n_A = 81$ and $n_B = 33$, leading to a profit of 408\$.

        \item By doubling the production capacity without modifying the other constraints, the resulting profit is doubled. The company should then pay less than 408\$ for renting an extra machine thath is profitable.
\end{enumerate}


\end{document}
