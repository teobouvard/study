\documentclass[a4paper, 10pt, twoside]{article}
\usepackage[left=2cm, right=2cm, top=2cm, bottom=3cm]{geometry}
\usepackage{amsmath}
\usepackage[shortlabels]{enumitem}
\usepackage{bbold}
\usepackage{cases}
\usepackage{systeme}
\usepackage{graphicx}

\begin{document}



\title{Algorithm Theory - Assignment 1}
\author{T\'eo Bouvard}
\maketitle

\section*{Problem 1}
In this problem, the objective is to maximize the profit of selling products X and Y, while satisfying production constraints on these products. The profit can be computed as

\begin{align*}
    profit & = revenue - cost                                                                              \\
           & = revenue - (time_{machine} \times cost_{machine} + time_{craftsman} \times cost_{craftsman})
\end{align*}

We can compute profits for each of the products

\begin{align*}
    profit(X) = 200 - (\frac{15}{60} \times 100 + \frac{20}{60} \times 20) = \frac{505}{3} \\
    profit(Y) = 300 - (\frac{20}{60} \times 100 + \frac{30}{60} \times 20) = \frac{770}{3}
\end{align*}

And formulate the problem as a Linear Programming problem. Let $n_X$ and $n_Y$ be the number of products X and Y produced.

\begin{align*}
     & \text{maximize } n_X \times profit(X) + n_Y \times profit(Y) \\
     & \text{subject to }
    \begin{cases}
        n_X \times time_{machine}(X) + n_Y \times time_{machine}(Y) \le 40 \times 60     \\
        n_X \times time_{craftsman}(X) + n_Y \times time_{craftsman}(Y) \le 35 \times 60 \\
        n_X \ge 10                                                                       \\
        n_X, n_Y \ge 0                                                                   \\
    \end{cases}
\end{align*}

Which can be simplified as

\begin{align*}
     & \text{maximize } n_X \times \frac{505}{3} + n_Y \times \frac{770}{3} \\
     & \text{subject to }
    \begin{cases}
        15 \times n_X + 20 \times n_Y \le 2400 \\
        20 \times n_X + 30 \times n_Y \le 2100 \\
        n_X \ge 10                             \\
        n_Y \ge 0                              \\
    \end{cases}
\end{align*}

\section*{Problem 2}
\begin{enumerate}[a)]
    \item

          Let $n_{A}$ and $n_{B}$ be the number of products A and B produced.

          \begin{align*}
               & \text{maximize } n_{A} \times 3 + n_{B} \times 5 \\
               & \text{subject to }
              \begin{cases}
                  12 \times n_{A} + 25 \times n_{B} \le 30 \times 60 \\
                  2 \times n_{B} - 5 \times n_{A} \ge 0              \\
                  n_{A}, n_{B} \ge 0                                 \\
              \end{cases}
          \end{align*}

          \begin{center}
              \includegraphics[width = .5 \textwidth]{graph1.png}
          \end{center}

          By graphing the feasible region, we see three candidate points. The first one $I_0 = (0, 0)$ can be trivially discarded as it leads to a profit of 0\$. Let's evaluate the objective function at the two other points. At $I_1 = (0, 72)$ the profit is 216\$. To compute the coordinates of the last point, we solve the following system.

          %\begin{align*}
          %    \systeme{
          %        12 \times n_{A} + 25 \times n_{B} = 1800,
          %        2 \times n_{B} - 5 \times n_{A} = 0
          %    }
          %    \implies
          %    \systeme{
          %        22 \times n_{A} = 1800,
          %        n_{B} = \frac{2}{5} \times n_{A}
          %    }
          %    \implies
          %    \systeme{
          %        n_{A} = 81.8,
          %        n_{B} = 32.7
          %    }
          %\end{align*}

          and at $I_2 = (24.2, 60.4)$ the profit is 374.5\$ which is the highest profit in this case. However, $n_A$ and $n_B$ are not integers. It is not specified in the exercise whether they should be or not, but as the problem is about a weekly production, we can assume that they do not need to be integer quantities as the weeks are continuous. If we wanted to restrict the problem to an integer solution, we should pick $n_A = 24$ and $n_B = 60$, leading to a profit of 372\$.

    \item By doubling the production capacity without modifying the other constraints, the resulting profit is doubled. The company should then pay less than 374.5\$ for renting an extra machine in order to stay profitable.
\end{enumerate}

\section*{Problem 3}

\begin{enumerate}[a)]
    \item We want to minimize the total cost of transportation, while respecting constraints on the flights. This problem can be formulated as the following linear programming problem.
          Let $n_{A}$ and $n_{B}$ be the number of flights flown with aircrafts A and B respectively.

          \begin{align*}
               & \text{minimize } n_{A} \times 10000 + n_{B} \times 12000 \\
               & \text{subject to }
              \begin{cases}
                  30 \times n_{A} + 15 \times n_{B} \ge 300    \\
                  500 \times n_{A} - 750 \times n_{B} \ge 9000 \\
                  n_{A} + n_{B} \le 16                         \\
                  n_{A}, n_{B} \ge 0                           \\
              \end{cases}
          \end{align*}

          \begin{center}
            \includegraphics[width = .5\textwidth]{graph2.png}
          \end{center}

          With the graph method, we can identify the three points delimiting the feasible region. We first determine their coordinates, and then evaluate the objective function at these points.

          \begin{align*}
              I_1 :
              \systeme{
                  30x+15y=300,
                  x+y=16
              }
              \implies
              \systeme[.]{
                  x=4,
                  y=12
              }
              \implies \text{cost } = 184000\,\$ \\
              I_2 :
              \systeme{
                  30x+15y=300,
                  500x+750y=9000
              }
              \implies
              \systeme[.]{
                  x=6,
                  y=8
              }
              \implies \text{cost } = 156000\,\$ \\
              I_3 :
              \systeme{
                  x+y=16,
                  500x+750y=9000
              }
              \implies
              \systeme[.]{
                  x=12,
                  y=4
              }
              \implies \text{cost } = 168000\,\$ \\
          \end{align*}

          The lowest cost is achieved by using 6 flights with A and 8 flights with B, for a total cost of 156000\$.

\end{enumerate}

\section*{Problem 4}

\begin{enumerate}[a)]
    \item Graph method
    \begin{center}
        \includegraphics[width = .5 \textwidth]{graph3.png}
    \end{center}

          \begin{align*}
              I_1 :
              \systeme[]{
                  x_1 = 0,
                  x_2 = 0
              }
              \implies \text{objective } = 0                             \\
              I_2 :
              \systeme[]{
                  x_1 = 30,
                  x_2 = 0
              }
              \implies \text{objective } = 90                            \\
              I_3 :
              \systeme[]{
                  x_1 = 0,
                  x_2 = 25
              }
              \implies \text{objective } = 125                           \\
              I_4 :
              \systeme{
                  x_1+2x_2=50,
                  8x_1+3x_2=240
              }
              \implies
              \systeme[]{
                  x_1 = \frac{330}{13},
                  x_2 = \frac{160}{13}
              }
              \implies \text{objective } = \frac{1790}{13} \approx 137.7 \\
          \end{align*}

    \item Simplex algorithm

          We first convert the linear problem to its the slack form by introducing two slack variables $s_1$ and $s_2$.

          \begin{align*}
               & \text{maximize } z = 3x_1 + 5x_2 \\
               & \text{subject to }
              \begin{cases}
                  s_1 = 50 - x_1 - 2x_2    \\
                  s_2 = 240 - 8x_1 - 3x_2  \\
                  x_1, x_2, s_1, s_2 \ge 0 \\
              \end{cases}
          \end{align*}

          A basic feasible solution is $(x_1, x_2, s_1, s_2) = (0, 0, 50, 240)$. However, this solution is not optimal as the objective function can be improved by picker a greater $x_1$ or $x_2$. To do the first pivot, we choose $x_2$ as the entering variable because it has the highest coefficient in the objective function. To choose the leaving variable, we find the tightest constraint on $x_2$. In the first constraint, $x_2$ is limited to 25, and in the second constraint it is limited to 80. Thus the leaving variable is $s_1$, and we rewrite the problem with $x_2 = 25 - \frac{x_1}{2} - \frac{s_1}{2}$

          \begin{align*}
               & \text{maximize } z = 125 + \frac{x_1}{2} - \frac{5}{2}s_1 \\
               & \text{subject to }s
              \begin{cases}
                  x_2 = 25 - \frac{x_1}{2} - \frac{s_1}{2}     \\
                  s_2 = 165 - \frac{13}{2}x_1 + \frac{3}{2}s_1 \\
                  x_1, x_2, s_1, s_2 \ge 0                     \\
              \end{cases}
          \end{align*}

          The only remaining entering variable able to increase the objective function is $x_1$. Constraint 1 limits it to 50 and constraint 2 limits it to $\frac{330}{13} \approx 25.4$. The second constraint being tighter than the first, we choose $s_2$ as leaving variable and rewrite the problem with $x_1=\frac{330}{13}+\frac{3}{10}s_1-\frac{2}{13}s_2$

          \begin{align*}
               & \text{maximize } z = \frac{1790}{13} - \frac{47}{20}s_1 - \frac{1}{26}s_2 \\
               & \text{subject to }s
              \begin{cases}
                  x_2 = \frac{160}{13}-\frac{8}{13}s_1+\frac{1}{13}s_2 \\
                  x_1=\frac{330}{13}+\frac{3}{10}s_1-\frac{2}{13}s_2   \\
                  x_1, x_2, s_1, s_2 \ge 0                             \\
              \end{cases}
          \end{align*}

          At this stage, the basic feasible solution of the objective function cannot be increased further because there is only substractions of positive terms. Therefore, the optimal solution is $z = \frac{1790}{13} \approx 137$, obtained for the variables $x_1 = \frac{330}{13}$ and $x_2 = \frac{160}{13}$.

          \item Programming method
\end{enumerate}

\end{document}
